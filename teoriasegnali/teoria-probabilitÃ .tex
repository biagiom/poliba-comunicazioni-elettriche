%il Capitolo 4 è stato revisionato dal Prof. Grieco il 12/1/2017
\chapter{Teoria della probabilità}
\label{ch:teoriasegnali-capitolo4}
\section{Concetti di base}
Si definisce \textsc{spazio campione}\index{spazio campione} $\Omega$ l'insieme di tutti i possibili risultati di un esperimento o eventi. Un \textsc{evento}\index{evento} è un qualsiasi sotto insieme di possibili risultati di un esperimento, sottoinsieme di $\Omega$.

Dato $A$, un evento, si definisce il suo complemento $\bar{A}$, ovvero tutto lo spazio campione meno A, $\bar{A}=\Omega\setminus A$

$A\cup B$ (unione di A e B) è il sottoinsieme degli eventi che appartengono ad A o a B

$A\cup\bar{A}=\Omega$ l'unione di A e il complemento di A costituisce l'intero spazio campione

$A\cap B$ (intersezione di A e B) è il sottoinsieme degli eventi che appartengono sia ad A che a B

$A\cap\bar{A}=\emptyset$ l'intersezione tra eventi complementari è l'insieme vuoto.

\section{Spazio di probabilità}
Una definizione assiomatica di uno spazio di probabilità per \textsc{esperimenti aleatori} è caratterizzata da
\begin{itemize}
	\item uno spazio campione $\Omega$
	\item un insieme $S$ degli eventi di interesse (non necessariamente $\Omega$)
	\item una legge di probabilità $P(.)$ secondo Kolmogorov
\end{itemize}

\section{Legge di probabilità}
Una \index{legge di probabilità}\textsc{legge di probabilità} secondo Kolmogorov mette in corrispondenza un evento $A\in S$ con un numero reale tale che
\begin{equation}
	P(A)\geq 0
\end{equation}
\begin{equation}
	P(\Omega)=1
\end{equation}
eventi mutualmente esclusivi
\begin{equation}
	A\cap B=0\implies P(A\cup B)=P(A)+P(B)
\end{equation}

\section{Modelli alternativi}
Modelli matematici alternativi di probabilità sono la definizione di Laplace secondo cui, per uno spazio campione finito e un esperimento simmetrico (esempio: lancio di un dado non truccato), la probabilità è calcolata come numero di eventi favorevoli fratto il numero di eventi possibili
\begin{equation}
	P(A)=\frac{N_F(A)}{N_P(A)}
\end{equation}

Definizione di Van Mises o \emph{frequentista} per cui ripetendo $N$ volte un esperimento e contando il numero $N_A$ di volte in cui si verifica l'evento A, si ha che la \textsc{frequenza relativa} $\frac{N_A}{N}$ si approssima al valore della probabilità di A
\begin{equation}
	P(A)=\lim\limits_{N\to\infty}\frac{N_A}{N}
\end{equation}

La definizione frequentista rientra nella teoria assiomatica infatti $P(A)=\frac{N_A}{N}\geq 0$, se $A=\Omega\implies N_A=N\implies P(A)=1$, inoltre e $A\cap B=\emptyset$ allora $N_{A\cup B}=N_A+N_B\implies P(A\cup B)= \lim\limits_{N\to\infty}\frac{N_{A\cup B}}{N}=\lim\limits_{N\to\infty}\frac{N_A+N_B}{N}=P(A)+P(B)$

\section{Teoremi}
A partire da tali assiomi si derivano i seguenti teoremi
\begin{equation}
	P(\bar{A})=1-P(A)
\end{equation}

\begin{proof}[Dim.]
$A\cup\bar{A}=\Omega\implies P(A\cup\bar{A})=1$

$A\cap\bar{A}=0\implies P(A\cup\bar{A})=P(A)+P(\bar{A})$
\end{proof}

\begin{equation}
	P(\emptyset)=0
\end{equation}

\begin{proof}[Dim.]
$\bar{\Omega}=\emptyset\implies P(\emptyset)=P(\bar{\Omega})=1-P(\Omega)=0$
\end{proof}

\begin{equation}
0\leq P(A)\leq 1
\end{equation}

\begin{proof}[Dim.]
Per il primo assioma $P(A)\geq 0$ e $P(\bar{A})\geq 0$ ma è anche $P(A)=1-P(\bar{A})$
\end{proof}

\subsection{Teorema probabilità congiunte}
\index{Teorema!probabilità congiunte}
\begin{equation}
	P(A\cup B)=P(A)+P(B)-P(A\cap B)
\end{equation}

\begin{proof}[Dim.]
\[
	A\cup B=A\cup(B\cap\bar{A})\implies P(A\cup B)=P(A)+P(B\cap\bar{A})
\]

\[
	B=B\cap(A\cup\bar{A})=(B\cap A)\cup(B\cap\bar{A})\implies P(B)=P(B\cap A)+P(B\cap\bar{A})
\]
insieme $\implies P(A\cup B)=P(A)+P(B)-P(B\cap A)$
\end{proof}

\subsection{Teorema probabilità condizionata}
\index{Teorema!probabilità condizionata}
$P(A)$ è la probabilità che si verifichi l'evento A a priori.
$P(A|B)$ è la probabilità che si verifichi l'evento A dato che si sia verificato l'evento B, la prob. di A a posteriori:
\begin{equation}
	P(A|B)=\frac{P(A\cap B)}{P(B)}
\end{equation}

Nel caso particolare che A e B siano \textsc{eventi indipendenti}\index{eventi indipendenti}, la probabilità che si verifichi A non è influenzata da B, pertanto
\[
	P(A|B)=P(A)\qquad P(A\cap B)=P(A)P(B)
\]

\subsection{Teorema di Bayes}
\index{Teorema!di Bayes}
\begin{equation}
	P(A|B)=\frac{P(B|A)P(A)}{P(B)}
\end{equation}
che si può verificare facilmente sostituendo alla prob. condizionata $P(B|A)=\frac{P(A\cap B)}{P(A)}$

\subsection{Teorema delle Probabilità Totali (Bayes)}
\index{Teorema!delle Probabilità Totali (Bayes)}
Data una partizione dello spazio campione $\bigcup_i B_i=\Omega, B_i\cap B_j=\emptyset \;\forall i\neq j$
\begin{equation}
	P(A)=\sum_i P(A|B_i)P(B_i)
\end{equation}

\begin{proof}[Dim.]
Si dimostra facilmente notando che $A=\bigcup_i B_i \cap A=\bigcup_i (A\cap B_i) \implies P(A)=\sum_i P(A\cap B_i)=\sum_i P(A|B_i)P(B_i)$
\end{proof}

\section{Esperimento aleatorio composto}
Si considerano due esperimenti aleatori differenti caratterizzati ognuno dal proprio spazio campione. Si può considerare un esperimento composto in cui si osservano i due eventi $A_1\subseteq\Omega_1$ e $A_2\subseteq\Omega_2$, la coppia $A=A_1\times A_2$ nel nuovo spazio campione $\Omega_1\times\Omega_2$.

I due eventi possono essere indipendenti, in tal caso la probabilità dell'evento composto è pari al prodotto delle probabilità $P(A)=P(A_1)\cdot P(A_2)$.

Se i due esperimenti sono in qualche modo legati, si influenzano, sarà necessario invece valutare il grado di correlazione dei due eventi.

Le medesime considerazioni fatte per due esperimenti si possono fare per la composizione di $N$ qualsiasi esperimenti o prove ripetute ed indipendenti in un esperimento composto.

\subsection{Prove binarie di Bernoulli}
Esperimento composto da $n$ prove ripetute di esperimenti uguali tra loro e indipendenti, ciascuno con due soli possibili esiti $\omega_0$ e $\omega_1$ con probabilità $P(\omega_0)=p$ e $P(\omega_1)=1-p$.
Esempio classico il lancio di $n$ monete o il lancio ripetuto di una stessa moneta con risultato la sequenza di testa o croce.

Si definisce l'evento $A=\lbrace\omega_0$ si presenta $k$ volte in $n$ esperimenti$\rbrace$, $0\leq k\leq n$

La \textsc{formula binomiale di Bernoulli} esprime la probabilità dell'evento A:
\begin{equation}
	P(A)=\binom{n}{k} p^k (1-p)^{n-k}
\end{equation}

Il coefficiente binomiale
\begin{equation}
	\binom{n}{k}=\frac{n!}{k!(n-k)!}
\end{equation}
tiene conto delle possibili \textsc{combinazioni} in cui $k$ elementi possono essere disposti in $n$ posizioni non distinguendoli per l'ordine. Si ottiene dal numero di \textsc{disposizioni} possibili di $k$ oggetti in $n$ posizioni distinguendo le posizioni $D_{n,k}=n\cdot(n-1)\dots(n-k+1)=\frac{n!}{(n-k)!}$ diviso il numero delle \textsc{permutazioni} $P_{n,k}=k!$ di $k$ oggetti in $n$ posizioni.

\section{Esempi esperimenti aleatori}
\begin{esempio}
Esperimento di estrazione di palline bianche e nere da 5 urne di 3 tipi:
\begin{itemize}
	\item $A_1$: 2 urne con 2 palline bianche 1 nera
	\item $A_2$: 1 urna con 10 palline nere
	\item $A_3$: 2 urne con 3 palline bianche 1 nera
\end{itemize}
Evento $E_1=\lbrace$ pallina estratta da urna a caso è bianca $\rbrace$
\[
	E_1=(A_1\cap W)\cup(A_2\cap W)\cup(A_3\cap W)
\]
Per il teorema delle probabilità totali
\[
	P(E_1)=P(A_1\cap W)+P(A_2\cap W)+P(A_3\cap W)=\frac{4}{15}+0+\frac{3}{10}
\]
dove per il teorema delle probabilità condizionate
\[
\begin{split}
	P(A_1\cap W)&=P(W|A_1) P(A_1)=\frac{2}{3}\cdot\frac{2}{5}=\frac{4}{15}\\
	P(A_2\cap W)&=P(W|A_2) P(A_2)=0\cdot\frac{1}{5}=0\\
	P(A_3\cap W)&=P(W|A_3) P(A_3)=\frac{3}{4}\cdot\frac{2}{5}=\frac{3}{10}
\end{split}
\]

Evento $E_2=\lbrace$ pallina estratta a caso sia bianca da urna di tipo 3 $\rbrace$
\[
	E_2=(A_1\cap W)\cup(A_2\cap W)\cup(A_3\cap W)
\]

Per il teorema di Bayes la probabilità condizionata è
\[
	P(E_2)=P(A_3|W)=\frac{P(W|A_3)P(A_3)}{P(W)}=\frac{\frac{3}{4}\cdot\frac{2}{5}}{\frac{17}{30}}=\frac{3}{4}\cdot\frac{2}{5}\cdot\frac{30}{17}=\frac{9}{17}\cong 0,529
\]
\end{esempio}

\begin{esempio}
Esperimento aleatorio composto
\begin{itemize}
	\item 1 scatola con 2000 articoli di cui 5\% grandi
	\item 1 scatola con 500 articoli di cui 40\% grandi
	\item 1 scatola con 1000 articoli di cui 10\% grandi
	\item 1 scatola con 1000 articoli di cui 10\% grandi
\end{itemize}
Si scelga una scatola a caso e si estragga un articolo a caso e si valuti la probabilità $P(G)$ che sia estratto un articolo grande.

Per il teorema delle probabilità totali ho
\[
	P(G)=\sum_{i=1}^{4}P(G|S_i)P(S_i)
\]

$P(S_i)=\frac{1}{4}$, $P(G|S_1)=0.05$, $P(G|S_2)=0.4$, $P(G|S_3)=0.1$, $P(G|S_4)=0.1$

\[
	P(G)=\frac{1}{4}(0.05+0.4+0.1+0.1)=\frac{1}{4}\cdot 0.65=0.1625
\]
Nel caso sia estratto un articolo grande, quale è la probabilità che sia dalla scatola 3?

Per il teorema di Bayes
\[P(S_3|G)=\frac{P(G|S_3)P(S_3)}{P(G)}=0.1\cdot\frac{1}{4}\cdot\frac{1}{0.1625}\cong 0.154\]
\end{esempio}

\begin{esempio}
Esperimento aleatorio composto

Ho 6 scatole di 3 tipi diversi ognuna con 12 palline alcune bianche alcune nere
\begin{itemize}
	\item $A_1$ 1 scatola con 7 palline bianche 5 nere
	\item $A_2$ 2 scatola con 5 palline bianche 7 nere
	\item $A_3$ 3 scatola con 4 palline bianche 8 nere
\end{itemize}
Ad ogni estrazione la pallina viene rimessa nella scatola.

Qual è la probabilità dell'evento $E=\lbrace 2B+1N \rbrace$ ?

Si può scomporre l'evento in eventi disgiunti ed applicare il teorema delle probabilità totali
\[
	P(E)=\sum_{i=1}^{3}P(E|A_i)P(A_i)
\]
dove data la ripartizione delle scatole $P(A_1)=\frac{1}{6}, P(A_2)=\frac{2}{6}=\frac{1}{3}, P(A_3)=\frac{3}{6}=\frac{1}{2}$
Le probabilità condizionate si calcolano come prove ripetute di Bernoulli binarie indipendenti sull'evento E:
\[
\begin{split}
	P(E|A_1)&=\binom{3}{2}P(B|A_1)^2 P(N|A_1)=3\left(\frac{7}{12}\right)^2\frac{5}{12}=\frac{5}{4}\left(\frac{7}{12}\right)^2\\
	P(E|A_2)&=\binom{3}{2}P(B|A_2)^2 P(N|A_2)=3\left(\frac{5}{12}\right)^2\frac{7}{12}=\frac{7}{4}\left(\frac{5}{12}\right)^2\\
	P(E|A_3)&=\binom{3}{2}P(B|A_3)^2 P(N|A_3)=3\left(\frac{4}{12}\right)^2\frac{8}{12}=\frac{2}{9}
\end{split}
\]

\[
\begin{split}
	P(E)&=P(E|A_1)P(A_1)+P(E|A_2)P(A_2)+P(E|A_3)P(A_3)=\\
	&=\frac{1}{6}\cdot\frac{5}{4}\left(\frac{7}{12}\right)^2+\frac{1}{3}\cdot\frac{7}{4}\left(\frac{5}{12}\right)^2+\frac{1}{2}\cdot\frac{2}{9}\cong 0.28
\end{split}
\]

Qual è la probabilità che si verifichi l'evento $E=\lbrace 2B+1N\rbrace$ da scatole del tipo $A_3$?
Per il teorema di Bayes
\[
	P(A_3|E)=\frac{P(E|A_3)P(A_3)}{P(E)}=\frac{2}{9}\cdot\frac{1}{2}\cdot\frac{1}{0.28}\cong 0,39
\]
\end{esempio}
